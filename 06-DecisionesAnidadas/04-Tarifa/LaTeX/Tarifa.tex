\documentclass{article}
\usepackage[pdfencoding=auto]{hyperref}

\begin{document}

\vspace{0.5cm}
\textbf{c. Demo interactivo}\\
\indent{\scriptsize(requiere de un lector de PDF con soporte JavaScript -ej: lea el PDF en el navegador-)}\\

{\footnotesize
  \begin{Form}
    \texttt{Valor cobrado: \TextField[name=valorCobrado,width=4cm]{}}

    \vspace{0.2cm}

    \texttt{Valor del taxímetro: \TextField[name=valorTaximetro,width=4cm]{}}

    \vspace{0.2cm}

    \texttt{\CheckBox[name=esEspecial]{Es domingo, festivo o nocturno (s/n): }}
    
    \vspace{0.2cm}

    \texttt{Destino:  \ChoiceMenu[name=destinoTrayecto,combo,width=4cm,edit=false,default=urbano]{}{urbano, lugar periférico, extra urbano, vía al aeropuerto}}
  
    \vspace{0.2cm}

    \TextField[name=salida,width=10cm,height=2.8cm,multiline=true,readonly=true]{}
    \end{Form}
    
    \vspace{0.2cm} 
     
    \begin{Form}
    \PushButton[onclick={
         URBANO               = `urbano`;
         PERIFERIA            = `lugar perif\\xe9rico`;
         EXTRA_URBANO         = `extra urbano`;
         VIA_AEROPUERTO       = `v\\xeda al aeropuerto`;
         VALOR_MINIMO         = 6000;
         VALOR_ESPECIAL       = 1300;
         VALOR_PERIFERIA      = 3400;
         VALOR_EXTRA_URBANO   = 4900;
         VALOR_VIA_AEROPUERTO = 1600;
        function main() {
          let valorCobrado = ingresarEntero('valorCobrado');      
          let valorTaximetro = ingresarEntero('valorTaximetro');
          let esEspecial = ingresarLogico('esEspecial');
          let destinoTrayecto = ingresarOpcion('destinoTrayecto');
          let tarifaReal = calcularTarifa(valorTaximetro, esEspecial, destinoTrayecto);
          let mensajeCobro = determinarMensajeCobro(valorCobrado, tarifaReal);
          let informeCobro = generarInformeCobro(valorCobrado, tarifaReal, mensajeCobro);
          mostrarMensaje(informeCobro);
        }
        function ingresarTexto(componente) {
            return this.getField(componente).value;
          }
        function ingresarEntero(componente) {          
            let entero = parseInt(ingresarTexto(componente));
            if (isNaN(entero)) {
              mostrarError(`El valor ingresado es inv\\xe1lido, se asume 0`);
              entero = 0;
            }
            return entero;
          }
        function mostrarMensaje(mensaje) {
						this.getField("salida").value = mensaje;
					}
        function mostrarError(mensaje) {
          app.alert(mensaje);
        }
				function ingresarLogico(componente) {
						return this.getField(componente).isBoxChecked(0);
					}
        function ingresarOpcion(componente) {
          return this.getField(componente).value;
        }
        function calcularTarifa(valorTaximetro, esEspecial, destinoTrayecto) {
          return determinarTarifaMinima(valorTaximetro) + determinarTarifaEspecial(esEspecial) + determinarTarifaDestino(destinoTrayecto);
        }
        function determinarTarifaMinima(valorTaximetro)  {
          return Math.max(valorTaximetro, VALOR_MINIMO);
        }
        function determinarTarifaEspecial(esEspecial) {
          let tarifaEspecial = 0;
          if (esEspecial) {
            tarifaEspecial = VALOR_ESPECIAL;
          }
          return tarifaEspecial;
        }
        function determinarTarifaDestino(destinoTrayecto) {
          let tarifaDestino = 0;
          if (destinoTrayecto == PERIFERIA) {
            tarifaDestino = VALOR_PERIFERIA;
          } else {
            if (destinoTrayecto == EXTRA_URBANO) {
              tarifaDestino = VALOR_EXTRA_URBANO;
            } else {
              if (destinoTrayecto == VIA_AEROPUERTO) {
                tarifaDestino = VALOR_VIA_AEROPUERTO;
              }
            }
          }
          return tarifaDestino;
        }
        function determinarMensajeCobro(valorCobrado, tarifaReal){
          let mensajeCobro = `JUSTO`;
          if (valorCobrado != tarifaReal) {
            mensajeCobro = `INJUSTO`;
          }
          return mensajeCobro;
        }
        function generarInformeCobro(valorCobrado, tarifaReal, mensajeCobro) {
          let cobrado = valorCobrado.toString().padStart(7, ` `);
          let tarifa = tarifaReal.toString().padStart(7, ` `);
          return  `\\nValor cobrado \\t$${cobrado}` +
                  `\\nValor real \\t$${tarifa}\\n` +
                  `\\nPor lo anterior el cobro es ${mensajeCobro}\\n`;
        }
        main();
        }]{\texttt{\textbf{Generar informe de cobro}}}
\end{Form}
}

\end{document}