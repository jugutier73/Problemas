\documentclass{article}
\usepackage[pdfencoding=auto]{hyperref}

\begin{document}

\vspace{0.5cm}
\textbf{c. Demo interactivo}\\
\indent{\scriptsize(requiere de un lector de PDF con soporte JavaScript -ej: lea el PDF en el navegador-)}\\

{\footnotesize
  \begin{Form}
    \texttt{Ingrese un texto con los mensajes a analizar:}
    
    \TextField[name=texto,width=10cm,height=2.8cm,multiline=true]{}
    \vspace{0.2cm}

    \TextField[name=salida,width=10cm,height=1.8cm,multiline=true,readonly=true]{}
    \end{Form}
    
    \vspace{0.2cm} 
     
    \begin{Form}
    \PushButton[%
    onclick={
    SIMBOLO_INCLUSIVO_1 = 'x';
    SIMBOLO_INCLUSIVO_2 = '@';
    function main() {
	    let texto = ingresarTexto('texto');
        let cantidadSimbolosInclusivos = contarEmpleoSimbolos(texto);
        let reporteSimbolos = generarReporteInclusivo(cantidadSimbolosInclusivos);
        mostrarMensaje(reporteSimbolos);
    }
    function mostrarMensaje(mensaje) {
		this.getField("salida").value = mensaje;
	}
    function ingresarTexto(componente) {
        let texto = this.getField(componente).value;
        return texto;
    }
    function contarEmpleoSimbolos(texto) {
       let cantidadSimbolosInclusivos = 0;
       for (caracter of texto) {
          if (caracter === SIMBOLO_INCLUSIVO_1 || caracter === SIMBOLO_INCLUSIVO_2) {
             cantidadSimbolosInclusivos += 1;
          }
       }
       return cantidadSimbolosInclusivos;
    }
    function generarReporteInclusivo(cantidadSimbolosInclusivos) {
	    return  `\\nSe emplearon ${cantidadSimbolosInclusivos} veces ` +
		        `los s\\xedmbolos inclusivos \\"${SIMBOLO_INCLUSIVO_1}\\" y `+
		        `\\"${SIMBOLO_INCLUSIVO_2}\\".`;
    }
    main();
    }]{Generar reporte de lenguaje inclusivo}
  \end{Form}
}

\end{document}