\documentclass{article}
\usepackage[pdfencoding=auto]{hyperref}

\begin{document}

\vspace{0.5cm}
\textbf{c. Demo interactivo}\\
\indent{\scriptsize(requiere de un lector de PDF con soporte JavaScript -ej: lea el PDF en el navegador-)}\\

{\footnotesize
  \begin{Form}
    \texttt{Ingrese la contraseña a analizar:}
    \TextField[name=contrasenia,width=4cm]{}
    \vspace{0.2cm}

    \TextField[name=salida,width=10cm,height=2.8cm,multiline=true,readonly=true]{}
    \end{Form}
    
    \vspace{0.2cm} 
     
    \begin{Form}
    \PushButton[%
    onclick={
    function main() {
        let contrasenia = ingresarTexto('contrasenia'); 
        let cantidadMayusculas = contarMayusculas(contrasenia);
        let cantidadMinusculas = contarMinusculas(contrasenia);
        let cantidadDigitos = contarDigitos(contrasenia);
        let reporteEstres = generarReporteContrasenia(contrasenia, cantidadMayusculas,cantidadMinusculas, cantidadDigitos);
        mostrarMensaje(reporteEstres);
    }
    function mostrarMensaje(mensaje) {
		this.getField("salida").value = mensaje;
	}
    function ingresarTexto(componente) {
        let texto = this.getField(componente).value;
        return texto;
    }
    function contarCaracteres(texto, funcion) {
        let cantidadCaracteres = 0;
        for (let i = 0 ; i < texto.length ; i++) {
            if (funcion(texto.charAt(i))) {
                cantidadCaracteres += 1
            }
        }
        return cantidadCaracteres;
    }
    UPPER = /\\p{Lu}/u;
    LOWER = /\\p{Ll}/u;
    DIGIT = /\\p{Nd}/u;
    esMayuscula  = caracter => UPPER.test(caracter);
    esMinuscula  = caracter => LOWER.test(caracter);
    esDigito     = caracter => DIGIT.test(caracter);
    contarMayusculas  = texto => contarCaracteres(texto, esMayuscula);
    contarMinusculas  = texto => contarCaracteres(texto, esMinuscula);
    contarDigitos     = texto => contarCaracteres(texto, esDigito);
    function generarReporteContrasenia(contrasenia,
        cantidadMayusculas, cantidadMinusculas, cantidadDigitos) {
        return `\\nEn la constrase\\xf1a \\"${contrasenia}\\"  hay:\\n ` +
            `${cantidadMayusculas} may\\xfascula(s), `+
            `${cantidadMinusculas} min\\xfascula(s) y `+
            `${cantidadDigitos} d\\xedgito(s)`;
            }
    main();
    }]{Generar reporte de estrés}
  \end{Form}
}

\end{document}