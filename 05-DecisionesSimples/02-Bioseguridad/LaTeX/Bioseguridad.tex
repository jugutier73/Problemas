\documentclass{article}
\usepackage{hyperref}

\begin{document}

\vspace{0.5cm}
\textbf{c. Demo interactivo}\\
\indent{\scriptsize(requiere de un lector de PDF con soporte JavaScript -ej: lea el PDF en el navegador-)}\\

{\footnotesize
  \begin{Form}
    \texttt{\CheckBox[name=tieneVacunas]{Tiene vacunas}}
  
    \texttt{\CheckBox[name=restadosNegativos]{Tiene resultados negativos en pruebas}}
  
    \texttt{\CheckBox[name=tieneSintomas]{Tiene síntomas}}
    
    \vspace{0.2cm}
  
    \TextField[name=salida,width=10cm,height=2.8cm,multiline=true,readonly=true]{}
    \end{Form}
    
    \vspace{0.2cm}
    
    \begin{Form}
    \PushButton[onclick={ 
        function main() {
             let tieneVacunas = ingresarLogico('tieneVacunas');
             let restadosNegativos = ingresarLogico('restadosNegativos');
             let tieneSintomas = ingresarLogico('tieneSintomas');
             let reporteIngreso = generarReporteIngreso(tieneVacunas, restadosNegativos, tieneSintomas);  
             mostrarMensaje(reporteIngreso);
          }
          function mostrarMensaje(mensaje) {
            this.getField("salida").value = mensaje;
          }
          function ingresarLogico(componente) {
            return this.getField(componente).isBoxChecked(0);
          }
          function generarReporteIngreso(tieneVacunas, restadosNegativos, tieneSintomas) {
            let mensaje = `\\nLa persona no puede ingresar al evento`;          
            if (tieneVacunas && restadosNegativos && !tieneSintomas){
              mensaje = `\\nLa persona puede ingresar al evento`;          
             }
            return mensaje;
          }     
        main();
    }]{\texttt{\textbf{Generar Reporte Ingreso}}}
    \end{Form}
} 

\end{document}