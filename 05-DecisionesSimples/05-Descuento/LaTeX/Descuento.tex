\documentclass{article}
\usepackage{hyperref}

\begin{document}

\vspace{0.5cm}
\textbf{c. Demo interactivo}\\
\indent{\scriptsize(requiere de un lector de PDF con soporte JavaScript -ej: lea el PDF en el navegador-)}\\

{\footnotesize
\begin{Form}
	\texttt{\CheckBox[name=esDomingoFestivo]{Es domingo o festivo}}

	\texttt{\CheckBox[name=esEstudiante]{Es estudiante}}

	\texttt{\CheckBox[name=esAdultoMayor]{Es adulto mayor}}

	\vspace{0.2cm}

	\TextField[name=salida,width=10cm,height=2.8cm,multiline=true,readonly=true]{}
\end{Form}

\vspace{0.2cm}

\begin{Form}
	\PushButton[name=generarReciboTarifa,onclick={
				% Se omite la instrucción (const), para evitar 
				% el error de intentar redefinirlas cada vez 
				% que se presiona el botón.
				TARIFA_NORMAL = 2900;
				TARIFA_DOMINGO_FESTIVO = 3000;
				PORCENTAJE_DESCUENTO_ESTUDIANTE = 10.0;
				PORCENTAJE_DESCUENTO_ADULTO_MAYOR = 15.0;
				function main() {
						let esDomingoFestivo = ingresarLogico('esDomingoFestivo');
						let esEstudiante = ingresarLogico('esEstudiante');
						let esAdultoMayor = ingresarLogico('esAdultoMayor');
						let tarifaDia = obtenerTarifaDia(esDomingoFestivo);
						let porcentajeDescuento = obtenerPorcentajeDescuento(esEstudiante, esAdultoMayor);
						let valorDescuento = calcularValorDescuento(tarifaDia, porcentajeDescuento);
						let valorTarifa = calcularValorTarifa(tarifaDia, valorDescuento);
						let reciboTarifa = generarReciboTarifa(tarifaDia, valorTarifa, porcentajeDescuento, valorDescuento);
						mostrarMensaje(reciboTarifa);
					}
				function mostrarMensaje(mensaje) {
						this.getField("salida").value = mensaje;
					}
				function ingresarLogico(componente) {
						return this.getField(componente).isBoxChecked(0);
					}
				function obtenerTarifaDia(esDomingoFestivo) {
						let tarifaDia = TARIFA_NORMAL;
						if (esDomingoFestivo == true) {
								tarifaDia = TARIFA_DOMINGO_FESTIVO;
							}
						return tarifaDia;
					}
				function obtenerPorcentajeDescuento(esEstudiante, esAdultoMayor) {
						let porcentajeDescuento = 0.0;
						if (esEstudiante == true) {
								porcentajeDescuento = PORCENTAJE_DESCUENTO_ESTUDIANTE;
							}
						if (esAdultoMayor == true) {
								porcentajeDescuento = PORCENTAJE_DESCUENTO_ADULTO_MAYOR;
							}
						return porcentajeDescuento;
					}
				function calcularValorDescuento(tarifaDia, porcentajeDescuento){
						return tarifaDia * porcentajeDescuento / 100.0;
					}
				function calcularValorTarifa(tarifaDia, valorDescuento) {
						return tarifaDia - valorDescuento;
					}
				function generarReciboTarifa(tarifaDia, valorTarifa, porcentajeDescuento, valorDescuento) {
						let mensaje = `\\nLa tarifa es de $${tarifaDia}`;
						if (valorDescuento > 0){
								mensaje = `${mensaje}` +
								`\\nsu tarifa a pagar es de $${valorTarifa.toFixed(0)} por tener `+
								`\\nun descuento del ${porcentajeDescuento.toFixed(1)}\\x25 `+
								`equivalente  a $${valorDescuento.toFixed(0)}`;
							}
						return mensaje;
					}
				main();
			}]{\texttt{\textbf{Generar Recibo Tarifa}}}
\end{Form}
}

\end{document}