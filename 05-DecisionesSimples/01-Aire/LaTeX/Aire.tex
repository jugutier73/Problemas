\documentclass{article}
\usepackage{hyperref}

\begin{document}

\vspace{0.5cm}
\textbf{c. Demo interactivo}\\
\indent{\scriptsize(requiere de un lector de PDF con soporte JavaScript -ej: lea el PDF en el navegador-)}\\

{\footnotesize
\begin{Form}
  \texttt{Índice calidad del aire:  \TextField[name=indiceCalidadAire,width=4cm]{}}
  
  \vspace{0.2cm}

  \TextField[name=salida,width=10cm,height=2.8cm,multiline=true,readonly=true]{}
  \end{Form}
  
  \vspace{0.2cm}
  
  \begin{Form}
  \PushButton[onclick={ 
      function main() {
          let indiceCalidadAire = ingresarReal('indiceCalidadAire');
          let alertaCalidadAire = generarAlertaCalidadAire(indiceCalidadAire);
          mostrarMensaje(alertaCalidadAire);
        }
        function mostrarMensaje(mensaje) {
          this.getField("salida").value = mensaje;
        }
        function mostrarError(mensaje) {
          app.alert(mensaje);
        }
        function ingresarTexto(componente) {
          return this.getField(componente).value;
        }
        function ingresarReal(componente) {
          let real = parseFloat(ingresarTexto(componente));
          if (isNaN(real)) {
            mostrarError(`El valor ingresado es inv\\xe1lido, se asume 0.0`);
            real = 0.0;
          }
          return real;
        }
        function generarAlertaCalidadAire(indiceCalidadAire) {
          let mensaje = `\\nEl aire supone un riesgo bajo para la salud`;
          if (indiceCalidadAire > 100.0) {
              mensaje = `\\nEl aire puede presentar efectos sobre la salud`;
          }
          return mensaje;
        }
        main();
  }]{\texttt{\textbf{Generar Alerta Calidad del Aire}}}
  \end{Form}
  } 

\end{document}