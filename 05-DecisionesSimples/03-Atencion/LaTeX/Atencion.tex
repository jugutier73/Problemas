\documentclass{article}
\usepackage{hyperref}

\begin{document}

\vspace{0.5cm}
\textbf{c. Demo interactivo}\\
\indent{\scriptsize(requiere de un lector de PDF con soporte JavaScript -ej: lea el PDF en el navegador-)}\\

{\footnotesize
  \begin{Form}
    \texttt{Edad del paciente:  \TextField[name=edadPaciente,width=4cm]{}}
  
    \texttt{\CheckBox[name=enfermedadCronica]{Enfermedad crónica (s/n):}}
  
    \texttt{\CheckBox[name=estadoInmunosupresion]{Estado de inmunosupresión (s/n): }}
    
    \vspace{0.2cm}
  
    \TextField[name=salida,width=10cm,height=2.8cm,multiline=true,readonly=true]{}
    \end{Form}
    
    \vspace{0.2cm}
    
    \begin{Form}
    \PushButton[onclick={
          EDAD_RECIEN_NACIDO = 1;
          EDAD_ADULTO_MAYOR = 60;
          function main() {
             let edadPaciente = ingresarEntero('edadPaciente');
             let enfermedadCronica = ingresarLogico('enfermedadCronica');
             let estadoInmunosupresion = ingresarLogico('estadoInmunosupresion');
             let reporteAtencion = generarReporteAtencion(edadPaciente, enfermedadCronica, estadoInmunosupresion);  
             mostrarMensaje(reporteAtencion);
          }
          function mostrarMensaje(mensaje) {
            this.getField("salida").value = mensaje;
          }
          function ingresarLogico(componente) {
            return this.getField(componente).isBoxChecked(0);
          }
          function mostrarError(mensaje) {
          app.alert(mensaje);
          }
          function ingresarTexto(componente) {
            return this.getField(componente).value;
          }
          function ingresarEntero(componente) {          
            let entero = parseInt(ingresarTexto(componente));
            if (isNaN(entero)) {
              mostrarError(`El valor ingresado es inv\\xe1lido, se asume 0`);
              entero = 0;
            }
            return entero;
          }
          function generarReporteAtencion(edadPaciente, enfermedadCronica, estadoInmunosupresion) {
            let mensaje = `\\nEl paciente es de atenci\\xf3n general`;          
            if (edadPaciente < EDAD_RECIEN_NACIDO || 
                edadPaciente > EDAD_ADULTO_MAYOR  || 
                enfermedadCronica || estadoInmunosupresion){
              mensaje = `\\nEl paciente es de atenci\\xf3n prioritaria`;          
             }
            return mensaje;
          }     
        main();
    }]{\texttt{\textbf{Generar reporte de atención}}}
    \end{Form}
} 

\end{document}