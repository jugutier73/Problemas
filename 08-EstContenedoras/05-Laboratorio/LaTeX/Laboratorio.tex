\documentclass{article}
\usepackage[pdfencoding=auto]{hyperref}

\begin{document}

\vspace{0.5cm}
\textbf{c. Demo interactivo}\\
\indent{\scriptsize(requiere de un lector de PDF con soporte JavaScript -ej: lea el PDF en el navegador-)}\\

{\footnotesize
    \begin{Form}
    \PushButton[%
    onclick={
      ANCHO = 20;
      ESTUDIANTE = 1;
      DOCENTE = 2;
      TECNICO = 3;
      SI = 4;
      ETIQUETAS = {
	      1: "Estudiante",
	      2: "Docente",
	      3: "Técnico"
      };
      function main() {
         let ingresos = ingresarColeccion(ingresarIngreso);
         let salidas = ingresarColeccion(ingresarSalidas);
         let personasLaboratorio = obtenerPersonasLaboratorio(ingresos, salidas);
         let cantidadEstudiantes = contarSegunCriterio(personasLaboratorio, tenerTipo, ESTUDIANTE);
         let cantidadDocentes = contarSegunCriterio(personasLaboratorio, tenerTipo, DOCENTE);
         let cantidadTecnicos = contarSegunCriterio(personasLaboratorio, tenerTipo, TECNICO);
         let reporteIngresos = generarReporteIngresos(
            ingresos,
            salidas,
            personasLaboratorio,
            cantidadEstudiantes,
            cantidadDocentes,
            cantidadTecnicos);
         mostrarMensaje(reporteIngresos);
      }
      function mostrarMensaje(mensaje) {
         this.getField("salida").value = mensaje;
      }
      function prompt(mensaje) {
         return app.response({cQuestion: mensaje});
      }
      function confirm(mensaje){
         return app.alert({cMsg: mensaje, nIcon: 2, nType: 2 }) == SI;
      }
      function ingresarColeccion(ingresarElemento) {
         const coleccion = [];
         let hayMas = true;
         while (hayMas) {
            coleccion.push(ingresarElemento());
            hayMas = confirm("Hay m\\xe1s datos (s:Ok / n:Cancelar): ");
         }
         return coleccion;
      }
      function ordenarColeccion(coleccion, comparador, descendente) {
         return [...coleccion].sort(
            (donante1, donante2) => descendente ? 
               comparador(donante2, donante1) :
               comparador(donante1, donante2)
         );
      }
      function convertirColeccionCadena(titulo, lista, convertirElementoCadena) {
         let mensaje = `\\n${titulo}\\n`;
         for (const elemento of lista) {
            mensaje += "\\t" + convertirElementoCadena(elemento);
         }
         return mensaje;
      }
      function contarSegunCriterio(coleccion, aplicarCriterio, valorCriterio) {
         return coleccion.filter((elemento) => aplicarCriterio(elemento, valorCriterio) ).length;
      }
      function ingresarIngreso() {
         app.alert("Ingrese los datos del ingreso:");
         let documento = prompt("Ingrese el documento   : ");
         let nombre = prompt("Ingrese el nombre      : ");
         let tipo = Number(prompt("TIPO DE PERSONAS AUTORIZADAS\\n" +
            "    1: Estudiante\\n" +
            "    2: Docente\\n" +
            "    3: T\\xe9cnico\\n" +
            "Ingrese tipo de persona      : ")) || 0;
         return { documento, nombre, tipo };
      }
      function ingresarSalidas() {
         app.alert("Ingrese los datos de la salida:");
         let documento = prompt("Ingrese el documento   : ");
         return documento;
      }
      function obtenerPersonasLaboratorio(ingresos, salidas) {
         salidasSet = new Set(salidas);
         return ingresos.filter(ing => !salidasSet.has(ing.documento));
      }
      function tenerTipo(ingresos, tipo) {
         return ingresos.tipo == tipo;
      }
      function generarReporteIngresos(ingresos,
         salidas,
         personasLaboratorio,
         cantidadEstudiantes,
         cantidadDocentes,
         cantidadTecnicos) {
         listadoIngresos = convertirColeccionCadena(
            "LISTADO DE INGRESOS",
            ingresos,
            convertirIngresoCadena);
         listadoSalidas = convertirColeccionCadena(
            "LISTADO DE SALIDAS",
            salidas,
            convertirSalidaCadena);
         listadoIngresos = convertirColeccionCadena(
            "LISTADO DE PERSONAS EN EL LABORATORIO",
            personasLaboratorio,
            convertirIngresoCadena);	
         return `${listadoIngresos}\\n\\n` +
            `${listadoSalidas}\\n\\n` +
            `${listadoIngresos}\\n\\n` +
            `Cantidad de Estudiantes : ${cantidadEstudiantes}\\n` +
            `Cantidad de Docentes    : ${cantidadDocentes}\\n` +
            `Cantidad de Tecnicos    : ${cantidadTecnicos}\\n`;
      }
      function convertirIngresoCadena(ingreso) {
         etiqueta = ETIQUETAS[ingreso.tipo];
         return `${ingreso.documento.padEnd(ANCHO)} ${ingreso.nombre.padEnd(ANCHO)} ${etiqueta}\\n`;
      }
      function convertirSalidaCadena(salida) {
         return salida;
      }      
      main();
    }]{Generar reporte recolección}
  \end{Form}

  \vspace{0.2cm} 

  \begin{Form}
     \TextField[name=salida,width=10cm,height=15cm,multiline=true,readonly=true]{}
  \end{Form} 

}

\end{document}