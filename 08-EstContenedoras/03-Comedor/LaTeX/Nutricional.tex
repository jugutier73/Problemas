\documentclass{article}
\usepackage[pdfencoding=auto]{hyperref}

\begin{document}

\vspace{0.5cm}
\textbf{c. Demo interactivo}\\
\indent{\scriptsize(requiere de un lector de PDF con soporte JavaScript -ej: lea el PDF en el navegador-)}\\

{\footnotesize
    \begin{Form}
    \PushButton[%
    onclick={
      SI = 4;
      function main() {
         let reservasComedor = ingresarColeccion(ingresarReverva);
         let cantidadConNecesidades = contarSegunCriterio  (reservasComedor, tenerNecesidadEspecial, true);
         mostrarMensaje("aaaa");
         let promedioEdades = calcularPromedioEdades(reservasComedor);
         let reporteReservas = generarReporteReservasComedor(
               reservasComedor,
               cantidadConNecesidades,
               promedioEdades);
         mostrarMensaje(reporteReservas);
      }
      function mostrarMensaje(mensaje) {
         this.getField("salida").value = mensaje;
      }
      function prompt(mensaje) {
         return app.response({cQuestion: mensaje});
      }
      function confirm(mensaje){
         return app.alert({cMsg: mensaje, nIcon: 2, nType: 2 }) == SI;
      }
      function ingresarColeccion(ingresarElemento) {
         const coleccion = [];
         let hayMas = true;
         while (hayMas) {
            coleccion.push(ingresarElemento());
            hayMas = confirm("Hay m\\xe1s datos (s:Ok / n:Cancelar): ");
         }
         return coleccion;
      }
      function ordenarColeccion(coleccion, comparador, descendente) {
         return [...coleccion].sort(
            (donante1, donante2) => descendente ? 
               comparador(donante2, donante1) :
               comparador(donante1, donante2)
         );
      }
      function convertirColeccionCadena(titulo, lista, convertirElementoCadena) {
         let mensaje = `\\n${titulo}\\n`;
         for (const elemento of lista) {
            mensaje += "\\t" + convertirElementoCadena(elemento);
         }
         return mensaje;
      }
      function contarSegunCriterio(coleccion, aplicarCriterio, valorCriterio) {
         return coleccion.filter((elemento) => aplicarCriterio(elemento, valorCriterio) ).length;
      }
      function ingresarReverva() {
         app.alert("Ingrese los datos de la reserva:");
         let nombre = prompt("Ingrese el nombre de la persona  :");
         let edad = Number(prompt("Ingrese la edad de la persona    :")) || 0;
         let necesidadEspecial = confirm("Tiene necesidades especiales (s:Ok / n:Cancelar): ");
         return { nombre, edad, necesidadEspecial };
      }
      function tenerNecesidadEspecial(reserva, necesidadEspecialInteres) {
         return reserva.necesidadEspecial == necesidadEspecialInteres;
      }
      function calcularPromedioEdades(reservasComedor) {
         let totalReservas = reservasComedor.length;
         let promedioEdades = 0.0;
         if ( totalReservas > 0) {
            promedioEdades = reservasComedor
               .reduce((sumaEdades, reserva) => sumaEdades + reserva.edad, 0) / totalReservas;
         }
         return promedioEdades;
      }
      function generarReporteReservasComedor(reservasComedor,
                                    cantidadConNecesidades,
                                    promedioEdades) {
         listadoReservas = convertirColeccionCadena(
               "LISTADO DE RESERVAS", 
               reservasComedor, 
               convertirReservaCadena);
         return `${listadoReservas}\\n\\n` +
            `Cantidad con necesidades : ${cantidadConNecesidades}\\n` +
            `Promedio de edades       : ${promedioEdades.toFixed(1)}\\n`;
      }
      function convertirReservaCadena(reserva) {
         let mensaje = `${reserva.nombre}, ${reserva.edad} a\\xf1os`;
         if (reserva.necesidadEspecial) {
            mensaje += ", con necesidad especial";
         }
         return mensaje + "\\n";
      }
      main();
    }]{Generar reporte reservas comedor}
  \end{Form}

  \vspace{0.2cm} 

  \begin{Form}
     \TextField[name=salida,width=10cm,height=15cm,multiline=true,readonly=true]{}
  \end{Form} 

}

\end{document}