\documentclass{article}
\usepackage[pdfencoding=auto]{hyperref}

\begin{document}

\vspace{0.5cm}
\textbf{c. Demo interactivo}\\
\indent{\scriptsize(requiere de un lector de PDF con soporte JavaScript -ej: lea el PDF en el navegador-)}\\

{\footnotesize
    \begin{Form}
    \PushButton[%
    onclick={
      NIVELES = 3;
      NIVEL_RIESGO      = 1;
      NIVEL_URGENCIA    = 2;
      NIVEL_PRIORITARIO = 3;
      SI = 4;
      function main() {
         let citas = ingresarColeccion(ingresarCita);
         let citasPorNivel = ordenarColeccion(citas, compararCampoNivel, false);
         let cantidadRiesgo      = contarSegunCriterio(citas, tenerNivelRiesgo, NIVEL_RIESGO);
         let cantidadUrgencia    = contarSegunCriterio(citas, tenerNivelRiesgo, NIVEL_URGENCIA);
         let cantidadPrioritario = contarSegunCriterio(citas, tenerNivelRiesgo, NIVEL_PRIORITARIO);
         let reporteCitas = generarReporteCitas(citasPorNivel,
                                                cantidadRiesgo,
                                                cantidadUrgencia,
                                                cantidadPrioritario);
         mostrarMensaje(reporteCitas);
      }
      function mostrarMensaje(mensaje) {
         this.getField("salida").value = mensaje;
      }
      function prompt(mensaje) {
         return app.response({cQuestion: mensaje});
      }
      function confirm(mensaje){
         return app.alert({cMsg: mensaje, nIcon: 2, nType: 2 }) == SI;
      }
      function ingresarColeccion(ingresarElemento) {
         const coleccion = [];
         let hayMas = true;
         while (hayMas) {
            coleccion.push(ingresarElemento());
            hayMas = confirm("Hay m\\xe1s datos (s:Ok / n:Cancelar): ");
         }
         return coleccion;
      }
      function ordenarColeccion(coleccion, comparador, descendente) {
         return [...coleccion].sort(
            (donante1, donante2) => descendente ? 
               comparador(donante2, donante1) :
               comparador(donante1, donante2)
         );
      }
      function convertirColeccionCadena(titulo, lista, convertirElementoCadena) {
         let mensaje = `\\n${titulo}\\n`;
         for (const elemento of lista) {
            mensaje += "\\t" + convertirElementoCadena(elemento);
         }
         return mensaje;
      }
      function ingresarCita() {
         app.alert("Ingrese los datos de la cita:");
         let nombre = prompt("Ingrese el nombre paciente:");
         let nivel = Number(prompt("NIVEL DE URGENCIA\\n"+
               "  1: Riesgo\\n"+
               "  2: Urgencia\\n"+
               "  3: Prioritario\\n"+
               "Ingrese tipo de persona: ")) || 0;
         return { nombre, nivel };
      }
      function compararCampoNivel(cita1, cita2) {
         return cita1.nivel - cita2.nivel;
      }
      function contarSegunCriterio(coleccion, aplicarCriterio, valorCriterio) {
         return coleccion.filter((elemento) => aplicarCriterio(elemento, valorCriterio) ).length;
      }
      function tenerNivelRiesgo(cita, nivelRiesgo) {
         return cita.nivel == nivelRiesgo
      }
      function generarReporteCitas(citasPorNivel,
                                 cantidadRiesgo,
                                 cantidadUrgencia,
                                 cantidadPrioritario) {
         listadoPorNivel = convertirColeccionCadena(
            "LISTADO POR NIVEL DE URGENCIA", 
            citasPorNivel, 
            convertirCitaCadena);
         return `${listadoPorNivel}\\n\\n` +
            `Nivel 1 (Riesgo)     : ${cantidadRiesgo}\\n` +
            `Nivel 2 (Urdencia)   : ${cantidadUrgencia}\\n` + 
            `Nivel 3 (Prioritario): ${cantidadPrioritario}\\n`;
      }
      function convertirCitaCadena(cita) {
         return `${cita.nivel} : ${cita.nombre}\\n`;
      }
      main();
    }]{Generar citas centro de salud}
  \end{Form}

  \vspace{0.2cm} 

  \begin{Form}
     \TextField[name=salida,width=10cm,height=15cm,multiline=true,readonly=true]{}
  \end{Form} 

}

\end{document}