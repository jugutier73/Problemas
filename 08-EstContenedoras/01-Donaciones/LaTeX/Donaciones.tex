\documentclass{article}
\usepackage[pdfencoding=auto]{hyperref}

\begin{document}

\vspace{0.5cm}
\textbf{c. Demo interactivo}\\
\indent{\scriptsize(requiere de un lector de PDF con soporte JavaScript -ej: lea el PDF en el navegador-)}\\

{\footnotesize
    \begin{Form}
    \PushButton[%
    onclick={
      UMBRAL = 10000;
      SI = 4;
      function main() {
         let donantes = ingresarColeccion(ingresarDonante);
         let donantesPorNombre     = ordenarColeccion(donantes, compararNombre,   false);
         let donantesPorOrdenacion = ordenarColeccion(donantes, compararDonacion, true);
         let mayoresDonantes = obtenerMayoresDonaciones(donantes, UMBRAL);
         let mayorDonante = obtenerMayorDonante(donantes);
         let sumaDonaciones = obtenerSumaDonaciones(donantes);
         let reporteDonaciones = generarReporteDonaciones(
            donantesPorNombre,
            donantesPorOrdenacion, 
            mayoresDonantes, 
            mayorDonante, 
            sumaDonaciones);
         mostrarMensaje(reporteDonaciones);
      }
      function mostrarMensaje(mensaje) {
         this.getField("salida").value = mensaje;
      }
      function prompt(mensaje) {
         return app.response({cQuestion: mensaje});
      }
      function confirm(mensaje){
         return app.alert({cMsg: mensaje, nIcon: 2, nType: 2 }) == SI;
      }
      function ingresarColeccion(ingresarElemento) {
         const coleccion = [];
         let hayMas = true;
         while (hayMas) {
            coleccion.push(ingresarElemento());
            hayMas = confirm("Hay m\\xe1s datos (s:Ok / n:Cancelar): ");
         }
         return coleccion;
      }
      function ingresarDonante() {
         app.alert("Ingrese los datos de un donante:");
         let nombre = prompt("Ingrese el nombre:");
         let donacion = Number(prompt("Ingrese donaci\\xf3n:")) || 0;
         return { nombre, donacion };
      }
      function ordenarColeccion(coleccion, comparador, descendente) {
         return [...coleccion].sort(
            (donante1, donante2) => descendente ? 
               comparador(donante2, donante1) :
               comparador(donante1, donante2)
         );
      }
      function compararNombre(d1, d2) {
         return d1.nombre.localeCompare(d2.nombre);
      }
      function compararDonacion(d1, d2) {
         return d1.donacion - d2.donacion;
      }
      function compararDonacion(d1, d2) {
      return d1.donacion - d2.donacion;
      }
      function obtenerMayoresDonaciones(coleccion, limite) {
         return coleccion.filter(donante => donante["donacion"] > limite);
      }
      function obtenerMayorDonante(coleccion) {
         return coleccion.reduce((donante1, donante2) =>
            donante1 == null ||
            donante2["donacion"] > donante1["donacion"] ?
            donante2 : donante1, null);
      }
      function obtenerSumaDonaciones(coleccion) {
         return coleccion.reduce(
            (sumaDonaciones, donante) => sumaDonaciones + donante["donacion"], 0
         );
      }
      function generarReporteDonaciones(donantesPorNombre, 
                                       donantesPorDonacion, 
                                       mayoresDonantes, 
                                       mayorDonante, 
                                       sumaDonaciones) {
         listadoPorNombre = convertirColeccionCadena(
               "LISTADO EN ORDEN ALFAB\\xc9TICO", 
               donantesPorNombre, 
               obtenerDonante);
         listadoPorDonacion = convertirColeccionCadena(
               "LISTADO ORDENADO POR DONACI\\xD3N", 
               donantesPorDonacion, 
               obtenerDonante);
         listadoMayores = convertirColeccionCadena(
               `LISTADO DONACIONES MAYORES A $${UMBRAL}`, 
               mayoresDonantes, 
               obtenerDonante);
         return `${listadoPorNombre}\\n${listadoPorDonacion}\\n${listadoMayores}\\n` +
            `El mayor donante: ${mayorDonante.nombre}\\n` +
            `Total de donantes $${sumaDonaciones}\\n`;
      }
      function convertirColeccionCadena(titulo, lista, convertirElementoCadena) {
         let mensaje = `\\n${titulo}\\n`;
         for (const elemento of lista) {
            mensaje += "\\t" + convertirElementoCadena(elemento);
         }
         return mensaje;
      }
      function obtenerDonante(donante) {
         return `$${String(donante.donacion).padStart(10)} \\t ${donante.nombre}\\n`;
      }
   main();
    }]{Generar reporte de donaciones}
  \end{Form}

  \vspace{0.2cm} 

  \begin{Form}
     \TextField[name=salida,width=10cm,height=15cm,multiline=true,readonly=true]{}
  \end{Form} 

}

\end{document}